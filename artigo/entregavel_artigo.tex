% !TeX program = pdfLaTeX
\documentclass[12pt]{article}
\usepackage{amsmath}
\usepackage{graphicx,psfrag,epsf}
\usepackage{enumerate}
\usepackage{natbib}
\usepackage{textcomp}
\usepackage[hyphens]{url} % not crucial - just used below for the URL
\usepackage{hyperref}
\providecommand{\tightlist}{%
  \setlength{\itemsep}{0pt}\setlength{\parskip}{0pt}}

%\pdfminorversion=4
% NOTE: To produce blinded version, replace "0" with "1" below.
\newcommand{\blind}{0}

% DON'T change margins - should be 1 inch all around.
\addtolength{\oddsidemargin}{-.5in}%
\addtolength{\evensidemargin}{-.5in}%
\addtolength{\textwidth}{1in}%
\addtolength{\textheight}{1.3in}%
\addtolength{\topmargin}{-.8in}%

%% load any required packages here




\setlength{\parindent}{1.5cm}
\usepackage{setspace}\doublespacing
\usepackage[brazilian]{babel}
\usepackage{csquotes}
\usepackage{float}
\usepackage{xcolor}
\floatplacement{figure}{H}

\begin{document}


\def\spacingset#1{\renewcommand{\baselinestretch}%
{#1}\small\normalsize} \spacingset{1}


%%%%%%%%%%%%%%%%%%%%%%%%%%%%%%%%%%%%%%%%%%%%%%%%%%%%%%%%%%%%%%%%%%%%%%%%%%%%%%

\if0\blind
{
  \title{\bf Comparação de modelos para a estimativa de volume de passageiros com
base na matriz origem-destino da telefonia móvel}

  \author{
        Fernando Antônio Pavão \\
    Instituto Tecnológico de Aeronáutica\\
     and \\     Lucas Coelho e Silva \\
    Instituto Tecnológico de Aeronáutica\\
      }
  \maketitle
} \fi

\if1\blind
{
  \bigskip
  \bigskip
  \bigskip
  \begin{center}
    {\LARGE\bf Comparação de modelos para a estimativa de volume de passageiros com
base na matriz origem-destino da telefonia móvel}
  \end{center}
  \medskip
} \fi

\bigskip
\begin{abstract}
The text of your abstract. 200 or fewer words.
\end{abstract}

\noindent%
{\it Keywords:} Modelo gravitacional, Machine Learning, Rede neural, Regressão linear múltipla
\vfill

\newpage
\spacingset{1.45} % DON'T change the spacing!

\hypertarget{introduuxe7uxe3o}{%
\section{Introdução}\label{introduuxe7uxe3o}}

A previsão de demanda de passageiros é um objeto de pesquisa
multidisciplinar cujos resultados podem auxiliar vários participantes do
mercado de transporte aéreo.

Do ponto de vista das linhas aéreas, a previsão de demanda de
passageiros é importante para a definição de rotas e frequências das
aeronaves. Para administradores de aeroportos, os resultados têm
aplicabilidade quanto à estimativa dos requisitos de infraestrutura, por
exemplo.

Existem diversas metodologias para estimar o volume de passageiros em
dado período de tempo. Na prática, é comum que a parte interessada
compare a previsão feita por mais de um modelo ao traduzir as
estimativas para uma decisão de negócio \citep{Grosche2007}. Dessa
forma, justifica-se um estudo que compare diferentes modelos para a
previsão de demanda.

Neste estudo, três modelos foram construídos e avaliados: o modelo
gravitacional, pioneiro e amplamente utilizado, e dois modelos de
aprendizagem de máquina: um de redes neurais, e o outro com a utilização
de regressão linear múltipla.

Os modelos foram construídos tendo como variáveis independentes dados
geoeconômicos, disponibilizados pelo \citet{atlasIpea}. O volume real de
passageiros utilizado é o disponível na matriz origem-destino construída
a partir dos dados de telefonia móvel, disponibilizados na base de dados
Hórus.

\hypertarget{modelos-de-previsuxe3o-de-demanda}{%
\section{Modelos de previsão de
demanda}\label{modelos-de-previsuxe3o-de-demanda}}

\hypertarget{modelo-gravitacional}{%
\subsection{Modelo gravitacional}\label{modelo-gravitacional}}

Os modelos gravitacionais foram os modelos causais pioneiros para a
previsão do volume de passageiros \citep{Grosche2007}. Sua utilizados
para explicar como o tráfego se distribui nos pares-cidade, indicando o
comportamento dos viajantes.

De acordo com \citet{Grosche2007}, a hipótese fundamental dos modelos
gravitacionais é a de que ele pode ser construído a partir de variáveis
econômicas ou sociais. Não obstante, essas variáveis podem ser de
natureza geoeconômica, como população e distribuição de renda, ou
relacionadas ao serviço de transporte aéreo em si, como preço de
passagens, ou conveniências disponíveis no aeroporto.

A equação clássica do modelo gravitacional é dada, para um par-cidade,
por:

\[T_{ij} = k V_i^\mu W_j^\alpha d_{ij}^{-\beta} \]

onde:

\begin{itemize}
\tightlist
\item
  \(T_{ij}\) é o fluxo entre a origem e o destino;
\item
  \(V_i\) é a \enquote{emissibilidade} da origem;
\item
  \(W_j\) é a \enquote{atratividade} do destino;
\item
  \(d_{ij}\) é a resistência ao fluxo - como a distância;
\item
  k é a constante de proporcionalidade;
\item
  \(\mu\), \(\alpha\) e \(\beta\) são os parâmetros do modelo.
\end{itemize}

A utilização do modelo gravitacional para a modelagem do volume de
passageiros no transporte aéreo é vasta na literatura.
\citet{Nommik2016} desenvolveram um modelo gravitacional para a
modelagem do volume de passageiros em rotas reginais, aplicado ao
Aeroporto de Tallinn (EETN). Na pesquisa, utilizaram variáveis
relacionadas ao serviço na construção do modelo, e demonstraram que o
modelo gravitacional pode ser uma ferramenta simples e efetiva para uma
análise primária conduzida por planejadores de rota na linha aérea.
\citet{Grosche2007} propuseram modelos gravitacionais uitilizando
variáveis geoeconômicas, visando a aplicabilidade do modelo em
pares-cidade que atualmente não possuam rotas estabelecidas. Os dois
modelos propostos apresentaram erro baixo e foram validados
estatisticamente.

\hypertarget{modelos-de-machine-learning}{%
\subsection{\texorpdfstring{Modelos de \emph{machine
learning}}{Modelos de machine learning}}\label{modelos-de-machine-learning}}

Embora menos utilizados que o modelo gravitacional para a modelagem de
volume de passageiros, encontram-se na literatura aplicações de modelos
de \emph{machine learning} nesse contexto.

\citet{pourebrahim2018} estudaram um modelo gravitacional e outro de
rede neural para a predição de fluxo de passageiros. Ainda que o foco do
estudo tenha sido a melhora das predições com a utilização de dados de
redes sociais, a comparação entre os modelos foi apresentada. De acordo
com os autores, em termos da Raiz do Erro Quadrático Médio (RMSE, do
inglês \emph{Root Mean Squared Error}), o modelo gravitacional
apresentou resultados melhores, enquanto o modelo de rede neural
apresentou um coeficiente de determinação (\(R^2\)) mais elevado.

\citet{erjongmanee2018} comparam modelos de regressão linear múltipla e
rede neural com o gravitacional para o volume de passageiros doméstico
na Tailândia. Tanto os modelos de regressão linear múltipla como os de
rede neural apresentaram erros baixos.

\hypertarget{desenvolvimento-dos-modelos}{%
\section{Desenvolvimento dos
modelos}\label{desenvolvimento-dos-modelos}}

\hypertarget{consolidauxe7uxe3o-dos-dados}{%
\subsection{Consolidação dos dados}\label{consolidauxe7uxe3o-dos-dados}}

Talvez seja benéfico tratar da parte comum e de decisões que afetem os
dois modelos em uma seção unificada. Exemplo: variáveis geoeconômicas
utilizadas.

\hypertarget{modelo-gravitacional-1}{%
\subsection{Modelo gravitacional}\label{modelo-gravitacional-1}}

Explicar as análises feitas para o desenvolvimento do modelo
gravitacional. Aqui, tratar da construção do modelo em si, calibragem
dos parâmetros, quantificação dos erros e validação da hipótese nula.

\hypertarget{modelos-de-machine-learning-1}{%
\subsection{\texorpdfstring{Modelos de \emph{machine
learning}}{Modelos de machine learning}}\label{modelos-de-machine-learning-1}}

Explicar as análises feitas para o desenvolvimento dos modelos de
machine learning.

\hypertarget{regressuxe3o-linear}{%
\subsubsection{Regressão linear}\label{regressuxe3o-linear}}

Inserir desenvolvimentos.

\hypertarget{rede-neural}{%
\subsubsection{Rede neural}\label{rede-neural}}

Inserir desenvolvimentos.

\begin{figure}[H]

{\centering \includegraphics[width=1\linewidth]{../figures/sample_figure_1} 

}

\caption{Figura de amostra}\label{fig:figuraExemplo}
\end{figure}

\hypertarget{conclusuxf5es}{%
\section{Conclusões}\label{conclusuxf5es}}

Conclusão dos resultados obtidos.

\bibliographystyle{agsm}
\bibliography{refs.bib}

\end{document}
