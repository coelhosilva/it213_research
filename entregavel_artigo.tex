% !TeX program = pdfLaTeX
\documentclass[12pt]{article}
\usepackage{amsmath}
\usepackage{graphicx,psfrag,epsf}
\usepackage{enumerate}
\usepackage{natbib}
\usepackage{textcomp}
\usepackage[hyphens]{url} % not crucial - just used below for the URL
\usepackage{hyperref}
\providecommand{\tightlist}{%
  \setlength{\itemsep}{0pt}\setlength{\parskip}{0pt}}

%\pdfminorversion=4
% NOTE: To produce blinded version, replace "0" with "1" below.
\newcommand{\blind}{0}

% DON'T change margins - should be 1 inch all around.
\addtolength{\oddsidemargin}{-.5in}%
\addtolength{\evensidemargin}{-.5in}%
\addtolength{\textwidth}{1in}%
\addtolength{\textheight}{1.3in}%
\addtolength{\topmargin}{-.8in}%

%% load any required packages here




\setlength{\parindent}{1.5cm}
\usepackage{setspace}\doublespacing
\usepackage[brazilian]{babel}
\usepackage{csquotes}
\usepackage{float}
\usepackage{xcolor}
\floatplacement{figure}{H}

\begin{document}


\def\spacingset#1{\renewcommand{\baselinestretch}%
{#1}\small\normalsize} \spacingset{1}


%%%%%%%%%%%%%%%%%%%%%%%%%%%%%%%%%%%%%%%%%%%%%%%%%%%%%%%%%%%%%%%%%%%%%%%%%%%%%%

\if0\blind
{
  \title{\bf Comparação de modelos para a estimativa de volume de passageiros com
base na matriz origem-destino da telefonia móvel}

  \author{
        Fernando Antônio Pavão \\
    Instituto Tecnológico de Aeronáutica\\
     and \\     Lucas Coelho e Silva \\
    Instituto Tecnológico de Aeronáutica\\
      }
  \maketitle
} \fi

\if1\blind
{
  \bigskip
  \bigskip
  \bigskip
  \begin{center}
    {\LARGE\bf Comparação de modelos para a estimativa de volume de passageiros com
base na matriz origem-destino da telefonia móvel}
  \end{center}
  \medskip
} \fi

\bigskip
\begin{abstract}
The text of your abstract. 200 or fewer words.
\end{abstract}

\noindent%
{\it Keywords:} Modelo gravitacional, Machine Learning
\vfill

\newpage
\spacingset{1.45} % DON'T change the spacing!

\hypertarget{introduuxe7uxe3o}{%
\section{Introdução}\label{introduuxe7uxe3o}}

A previsão de demanda de passageiros é um assunto multidisciplinar cujos
resultados podem auxiliar vários participantes do mercado de transporte
aéreo.

Do ponto de vista das linhas aéreas, a previsão de demanda de
passageiros pode ser utilizada para auxiliar na definição de novas
rotas. Para administradores de aeroportos, os resultados têm
aplicabilidade quanto à estimativa dos requisitos de infraestrutura, por
exemplo.

Existem diversas metodologias para a estimativa do volume de passageiros
em dado período de tempo. Na prática, é comum que a parte interessada
compare a previsão feita por mais de um modelo ao traduzir as
estimativas para uma decisão de negócio.

Dessa forma, justifica-se um estudo que compare diferentes modelos para
a previsão de demanda. Neste estudo, dois modelos foram construídos e
avaliados: o modelo gravitacional, pioneiro e amplamente utilizado, e um
modelo de aprendizagem de máquina (TODO: escolher qual modelo e citar
especificamente).

Os modelos foram construídos tendo como variáveis independentes dados
geoeconômicos (TODO: citar fonte do IBGE/IPEA). O volume real de
passageiros utilizado é o disponível na matriz origem-destino construída
a partir dos dados de telefonia móvel. (TODO: citar a fonte da base de
dados. Horus?).

\hypertarget{modelos-de-previsuxe3o-de-demanda}{%
\section{Modelos de previsão de
demanda}\label{modelos-de-previsuxe3o-de-demanda}}

\hypertarget{modelo-gravitacional}{%
\subsection{Modelo gravitacional}\label{modelo-gravitacional}}

Os modelos gravitacionais foram os modelos causais pioneiros para a
previsão de tráfego aéreo. Sua utilizados para explicar como o tráfego
se distribui nos pares-cidade, indicando o comportamento dos viajantes.

A hipótese fundamental dos modelos gravitacionais é a de que ele pode
ser construído a partir de variáveis econômicas ou sociais
\citep{Grosche2007}. Não obstante essas variáveis podem ser de natureza
geoeconômica, como população e distribuição de renda, ou relacionadas ao
serviço de transporte aéreo em si, como preço de passagens, ou
conveniências disponíveis no aeroporto.

\hypertarget{modelos-de-machine-learning}{%
\subsection{\texorpdfstring{Modelos de \emph{machine
learning}}{Modelos de machine learning}}\label{modelos-de-machine-learning}}

Explicar sobre os modelos de machine learning.

\hypertarget{desenvolvimento-dos-modelos}{%
\section{Desenvolvimento dos
modelos}\label{desenvolvimento-dos-modelos}}

\hypertarget{consolidauxe7uxe3o-dos-dados}{%
\subsection{Consolidação dos dados}\label{consolidauxe7uxe3o-dos-dados}}

Talvez seja benéfico tratar da parte comum e de decisões que afetem os
dois modelos em uma seção unificada.

\hypertarget{modelo-gravitacional-1}{%
\subsection{Modelo gravitacional}\label{modelo-gravitacional-1}}

Explicar as análises feitas para o desenvolvimento do modelo
gravitacional. Aqui, tratar da construção do modelo em si, calibragem
dos parâmetros, quantificação dos erros e validação da hipótese nula.

\hypertarget{modelos-de-machine-learning-1}{%
\subsection{\texorpdfstring{Modelos de \emph{machine
learning}}{Modelos de machine learning}}\label{modelos-de-machine-learning-1}}

Explicar as análises feitas para o desenvolvimento do modelo.

\hypertarget{conclusuxf5es}{%
\section{Conclusões}\label{conclusuxf5es}}

Conclusão dos resultados obtidos.

\bibliographystyle{agsm}
\bibliography{refs.bib}

\end{document}
