\documentclass[a4paper,9pt,twocolumn,twoside,]{pinp}

%% Some pieces required from the pandoc template
\providecommand{\tightlist}{%
  \setlength{\itemsep}{0pt}\setlength{\parskip}{0pt}}

% Use the lineno option to display guide line numbers if required.
% Note that the use of elements such as single-column equations
% may affect the guide line number alignment.

\usepackage[T1]{fontenc}
\usepackage[utf8]{inputenc}

% pinp change: the geometry package layout settings need to be set here, not in pinp.cls
\geometry{layoutsize={0.95588\paperwidth,0.98864\paperheight},%
  layouthoffset=0.02206\paperwidth, layoutvoffset=0.00568\paperheight}

\definecolor{pinpblue}{HTML}{185FAF}  % imagecolorpicker on blue for new R logo
\definecolor{pnasbluetext}{RGB}{101,0,0} %



\title{Comparação de modelos para a estimativa de volume de passageiros com
base na matriz origem-destino da telefonia móvel}

\author[a]{Fernando Antônio Pavão}
\author[a]{Lucas Coelho e Silva}

  \affil[a]{Departamento de Transporte Aéreo, Instituto Tecnológico de Aeronáutica}

\setcounter{secnumdepth}{5}

% Please give the surname of the lead author for the running footer
\leadauthor{}

% Keywords are not mandatory, but authors are strongly encouraged to provide them. If provided, please include two to five keywords, separated by the pipe symbol, e.g:
 \keywords{  Modelo gravitacional |  Machine Learning |  Rede neural |  Regressão linear múltipla  }  

\begin{abstract}
Your abstract will be typeset here, and used by default a visually
distinctive font. An abstract should explain to the general reader the
major contributions of the article.
\end{abstract}

\dates{This version was compiled on \today} 


% initially we use doi so keep for backwards compatibility
% new name is doi_footer
\doifooter{\url{https://dx.doi.org/prefix/suffix}}

\pinpfootercontents{IT-213}

\begin{document}

% Optional adjustment to line up main text (after abstract) of first page with line numbers, when using both lineno and twocolumn options.
% You should only change this length when you've finalised the article contents.
\verticaladjustment{-2pt}

\maketitle
\thispagestyle{firststyle}
\ifthenelse{\boolean{shortarticle}}{\ifthenelse{\boolean{singlecolumn}}{\abscontentformatted}{\abscontent}}{}

% If your first paragraph (i.e. with the \dropcap) contains a list environment (quote, quotation, theorem, definition, enumerate, itemize...), the line after the list may have some extra indentation. If this is the case, add \parshape=0 to the end of the list environment.

\acknow{Sample acknowledgements: this template package builds upon, and extends,
the work of the excellent
\href{https://cran.r-project.org/package=rticles}{rticles} package, and
both packages rely on the
\href{http://www.pnas.org/site/authors/latex.xhtml}{PNAS LaTeX} macros.
Both these sources are gratefully acknowledged as this work would not
have been possible without them.}

\hypertarget{introduuxe7uxe3o}{%
\section{Introdução}\label{introduuxe7uxe3o}}

Com 4,3 bilhões de passageiros e 58 milhões de toneladas de carga
transportados, 65 milhões de empregos diretos e indiretos, e um impacto
econômico global de 2.7 trilhões de dólares em 2019, a importância do
transporte aéreo para a sociedade e seu papel de facilitador para o
desenvolvimento econômico são evidentes \citep{ihlg_aviation_2019}.

Peça fundamental para o direcionamento adequado e dimensionamento das
necessidades do setor de transporte aéreo são os estudos de previsão de
demandas de passageiros, principalmente levando em conta o contexto de
constantes mudanças na sociedade e ciclos econômicos, aliados a um
mercado da aviação competitivo com margens de lucro estreitas
\citep{srisaeng_forecasting_2015}.

A previsão de demanda de passageiros é um objeto de pesquisa
multidisciplinar cujos resultados podem embasar decisões de vários
participantes do mercado da aviação civil. Para os Estados e entidades
formadoras de políticas públicas, esses estudos fundamentam o traçado de
planejamento estratégico da malha aeroportuária, e pautam necessidades e
investimentos no controle de tráfego aéreo, por exemplo. Do ponto de
vista das linhas aéreas, a previsão de demanda de passageiros é
importante para a definição de rotas, modelos e frequências das
aeronaves. Permitem, ainda, a quantificação dos efeitos dos serviços
oferecidos na quantidade de passageiros transportados. Para os
fabricantes de aeronaves, os estudos de demanda ajudam no direcionamento
da estratégia e escolha dos projetos de aeronaves a serem desenvolvidos.
Ademais, os efeitos dos estudos de demanda não se restringem às
entidades citadas, pautando também decisões nos fabricantes de motores
aeronáuticos, fornecedores, provedores de serviços, e outros
\citep{icao_doc_2006}.

Existem diversas metodologias para estimar a demanda de passageiros na
aviação. Em \citet{icao_doc_2006}, os métodos de previsão são divididos
em três categorias quanto a técnica: quantitativas, qualitativas, e
análises de decisão. A técnica quantitativa, ou matemática, é
subsequentemente dividida em análises de séries temporais e análises
causais - aquela que pressupõe uma relação de causa e efeito -, o foco
deste artigo.

\citet{icao_doc_2006} cita como exemplo de modelos causais os modelos
econométricos de regressão - interpretado também na literatura como um
modelo de \emph{machine learning} \citep{erjongmanee_air_2018} - e
modelos de distribuição de tráfego aéreo, sob os quais se encontra o
modelo gravitacional. Alternativamente, encontram-se também, na
literatura, aplicações modelos de aprendizagem de máquina -
principalmente o de redes neurais, como em
\citet{blinova_analysis_2007}, \citet{alekseev_multivariate_2009}, e
\citet{pourebrahim_enhancing_2018}.

Na prática, é comum que a parte interessada compare a previsão feita por
mais de um modelo ao traduzir as estimativas para uma decisão de negócio
\citep{grosche_gravity_2007}. Dessa forma, justifica-se um estudo que
compare diferentes modelos para a previsão de demanda.

A escolha por modelos causais está ancorada na base de dados utilizada.
Em 2020, a Secretaria Nacional de Aviação Civil, em parceria com o
Laboratório de Transportes e Logística da Universidade Federal de Santa
Catarina (UFSC) disponibilizaram ao público uma matriz origem destino -
incluindo de deslocamentos aéreos - de escala nacional no Brasil. Essa
matriz foi elaborada a partir de dados de telefonia móvel inéditos com o
objetivo de auxiliar na compreensão da origem verdadeira dos viajantes
pelo território nacional
\citep{coordenacao-geral_de_planejamento_pesquisas_e_estudos_da_aviacao_civil_matriz_2020}.
Dessa forma, desponta-se uma oportunidade de explorar esses novos dados
com a construção de modelos de demanda.

Neste estudo, três modelos foram construídos e avaliados: um modelo
gravitacional, um modelo econométrico de regressão linear múltipla e
outro de redes neurais, do campo de aprendizagem de máquinas. Os modelos
foram construídos tendo como variáveis independentes dados
geoeconômicos, disponibilizados pelo
\citet{instituto_de_pesquisa_economica_aplicada_atlas_2020}.

O artigo está organizado da seguinte maneira: a revisão de trabalhos
relacionados está apresentada na seção 2, seguida pela metodologia e
detalhamento dos modelos na seção 3. Os resultados são apresentados na
seção 4, e, as conclusões, na seção 5.

\hypertarget{revisuxe3o-bibliogruxe1fica}{%
\section{Revisão bibliográfica}\label{revisuxe3o-bibliogruxe1fica}}

\hypertarget{sec:modelos-demanda-rev}{%
\subsection{Modelos de previsão de
demanda}\label{sec:modelos-demanda-rev}}

\citet{icao_doc_2006} divide os modelos quantitativos para previsão de
demanda na aviação em duas grandes categorias: análises de séries
temporais, como análises de tendência, e métodos causais. Ainda segundo
a ICAO (\emph{International Civil Aviation Organization}), enquanto a
análise por séries temporais é considerada confiável no curto-prazo, a
provável imprecisão e dificuldade de embasamento teórico no longo prazo
leva a alternativas que considerem a influência das condições sociais,
operacionais e econômicas no desenvolvimento do tráfego aéreo
\citep{icao_doc_2006}.

Não obstante, um modelo de séries temporais encontraria dificuldades em
lidar com disrupções, como, por exemplo, a pandemia do COVID-19,
enquanto um modelo causal - especialmente o de regressão -, que leva em
consideração variáveis sociais e econômicas, está melhor fundamentado
teoricamente para quantificar a reação do tráfego
\citep{doganis_flying_2005}.

Do ponto de vista dos modelos causais, de acordo com
\citet{kanafani_transportation_1983}, a hipótese fundamental na
construção de um modelo de demanda de transporte é a de que a demanda
por transportes urbanos está diretamente relacionada com a demanda por
atividades urbanas, sendo aquela derivada destas. Isso reflete, conforme
\citet{kanafani_transportation_1983}, no costumeiro relacionamento
direto da demanda por transporte com características socioeconômicas e
atributos do sistema de transporte. \citet{jorge-calderon_demand_1997}
indica que a literatura define a demanda pelo transporte aéreo como
dependente de dois grupos de variáveis - as de natureza geoeconômicas,
como população e distribuição de renda, ou aquelas relacionadas ao
serviço de transporte aéreo em si, tendo como exemplos os preços de
passagens.

Alguns exemplos de modelos causais para a previsão de demanda
encontrados na literatura são o modelo gravitacional, modelos
econométricos - como o de regressão linear e o os de equações
simultâneas-, e modelos de aprendizagem de máquina.

\hypertarget{modelo-gravitacional}{%
\subsubsection{Modelo gravitacional}\label{modelo-gravitacional}}

Os modelos gravitacionais foram os modelos causais pioneiros para a
previsão do volume de passageiros \citep{grosche_gravity_2007}. Sua
utilizados para explicar como o tráfego se distribui nos pares-cidade,
indicando o comportamento dos viajantes.

Segundo \citet{grosche_gravity_2007}, uma das hipóteses do modelo
gravitacional é a de que ele pode ser construído a partir de variáveis
econômicas ou sociais, em consonância com os conceitos apresentados por
\citet{jorge-calderon_demand_1997} e
\citet{kanafani_transportation_1983} apresentados na seção
\ref{sec:modelos-demanda-rev}.

A equação clássica do modelo gravitacional é dada, para um par-cidade,
por:

\begin{equation}
T_{ij} = k V_i^\mu W_j^\alpha d_{ij}^{-\beta}
\end{equation}

onde \(T_{ij}\) é o fluxo entre a origem e o destino, \(V_i\) é a
``emissividade'' da origem, \(W_j\) é a ``atratividade'' do destino,
\(d_{ij}\) é a resistência ao fluxo - como a distância, por exemplo -, k
é a constante de proporcionalidade, e \(\mu\), \(\alpha\) e \(\beta\)
são os parâmetros do modelo.

A utilização do modelo gravitacional para a modelagem do volume de
passageiros no transporte aéreo é vasta na literatura.
\citet{nommik_developing_2016} desenvolveram um modelo gravitacional
para a modelagem do volume de passageiros em rotas regionais, aplicado
ao Aeroporto de Tallinn (EETN). Na pesquisa, utilizaram variáveis
relacionadas ao serviço na construção do modelo, e demonstraram que o
modelo gravitacional pode ser uma ferramenta simples e efetiva para uma
análise primária conduzida por planejadores de rota na linha aérea.
\citet{grosche_gravity_2007} propuseram modelos gravitacionais
utilizando variáveis geoeconômicas, visando a aplicabilidade do modelo
em pares-cidade que atualmente não possuam rotas estabelecidas. Os dois
modelos propostos apresentaram erro baixo e foram validados
estatisticamente.

Outras aplicações encontradas na literatura são
\citet{verleger_models_1972}, \citet{cohen_gravity_2016}, e
\citet{boelrijk_gravity_2019}, bem como em referências clássicas como
\citet{kanafani_transportation_1983}.

\hypertarget{modelo-economuxe9trico}{%
\subsubsection{Modelo econométrico}\label{modelo-economuxe9trico}}

Outra categoria de modelo causal comum e bastante referenciada na
literatura são os modelos econométricos. De acordo com a
\citet{icao_doc_2006}, um modelo econométrico busca explicar a demanda
de viagens aéreas - variável dependente - pelas mudanças nas variáveis
explanatórias, ou independentes. Exemplos de modelos econométricos são
os modelos de regressão linear ou os modelos de equações múltiplas.

Um exemplo clássico na literatura de aplicação de modelos econométricos
para estimar a demanda de passageiros no transporte aéreo é encontrado
em \citet{jorge-calderon_demand_1997}, aplicado ao tráfego aéreo
europeu. Nele, o autor constrói um modelo econométrico de equações
simultâneas, tendo como variáveis endógenas aquelas representantes da
frequência de voos, tamanho da aeronave e tarifa econômica. Ademais,
\citet{jorge-calderon_demand_1997} aplicou também variáveis
representativas da distância, população, renda, desconto de tarifas,
proximidade ou presença de \emph{hubs}, turismo, e se uma rota sobrevoa
o mar ou não. O modelo base proposto pelo autor apresentou um
coeficiente de determinação (\(R^2\)) de aproximadamente 0,95.

\hypertarget{modelos-de-machine-learning}{%
\subsubsection{\texorpdfstring{Modelos de \emph{machine
learning}}{Modelos de machine learning}}\label{modelos-de-machine-learning}}

Embora menos utilizados que os modelo gravitacional e econométrico para
a modelagem de volume de passageiros, encontram-se na literatura
aplicações de modelos de \emph{machine learning} nesse contexto. No
entanto, a maior parte dos modelos aplicados estão no escopo de análises
de séries temporais, ou como parte de um modelo causal envolvendo
modelos mais clássicos como o gravitacional.

\citet{blinova_analysis_2007} apresenta a aplicação de redes neurais com
o objetivo de prever a expansão da rede de transporte aéreo na Rússia,
no curto prazo. Com base em uma sequência de matrizes origem-destino -
uma para cada ano - a autora constrói redes do tipo TLFN
(\emph{Time-Lagged Feedforward Network}) em uma análise de séries
temporais.

Outra aplicação de redes neurais como modelo de previsão do transporte
aéreo é apresentada por \citet{alekseev_multivariate_2009}. Nesse
trabalho, os autores construíram um modelo de rede neural para previsão
da demanda, em termos de APK (\emph{passenger kilometers}), em uma série
temporal, adotando o cenário brasileiro como um estudo de caso. Segundo
\citet{alekseev_multivariate_2009}, o modelo de aprendizagem de máquina
desenvolvido superou o modelo econométrico tradicional.

Em \citet{erjongmanee_air_2018}, modelos de regressão linear múltipla e
rede neural derivados dos modelos gravitacionais foram desenvolvidos
para estimar o volume de passageiros doméstico na Tailândia. Tanto os
modelos de regressão linear múltipla como os de rede neural apresentaram
erros baixos.

A aplicação de um modelo de rede neural que mais se aproxima aos
objetivos deste trabalho é dada por \citet{pourebrahim_enhancing_2018}.
Os autores aplicaram dois modelos gravitacionais e dois modelos de rede
neural para a predição de fluxo de passageiros. Ainda que o foco do
estudo tenha sido a melhora das predições com a utilização de dados de
redes sociais, a comparação entre os modelos foi apresentada. Segundo
\citet{pourebrahim_enhancing_2018}, em termos da Raiz do Erro Quadrático
Médio (RMSE, do inglês \emph{Root Mean Squared Error}), o modelo
gravitacional apresentou resultados melhores, enquanto o modelo de rede
neural apresentou um coeficiente de determinação (\(R^2\)) mais elevado.

\hypertarget{matriz-origem-destino}{%
\subsection{Matriz origem-destino}\label{matriz-origem-destino}}

Um dos principais recursos para o planejamento de tráfego e estudos de
demanda é a matriz origem-destino (OD) representativa do tráfego entre
os pares-cidade.

O processo tradicional de construção de uma matriz OD consiste na
condução de pesquisas por amostra de domicílios. No entanto, uma das
grandes desvantagens do processo tradicional é o tempo necessário para a
obtenção da matriz OD e o custo envolvido
\citep{calabrese_estimating_2011}. Isso gera também uma dificuldade de
atualização dos dados \citep{fekih_data-driven_2020}.

Recentemente, métodos alternativos para a construção de matrizes de OD
vem sendo aplicados. A construção de matrizes origem-destino por meio de
dados da telefonia móvel, por exemplo, é um assunto atual e coberto na
literatura em estudos como \citet{calabrese_estimating_2011},
\citet{fekih_data-driven_2020}, \citet{friedrich_generating_2010} e
\citet{mellegard_origindestination-estimation_2011}.

No Brasil, uma iniciativa inédita da Secretaria Nacional de Aviação
Civil, em parceria com o Laboratório de Transportes e Logística da
Universidade Federal de Santa Catarina (UFSC), disponibilizou ao
público, em 2020, uma matriz OD de escala nacional, construída a partir
dos dados de telefonia móvel
\citep{coordenacao-geral_de_planejamento_pesquisas_e_estudos_da_aviacao_civil_matriz_2020},
e representativa dos deslocamentos por meios aéreos e não-aéreos. O
objetivo foi a compreensão dos fluxos verdadeiros dos viajantes no país.
Este artigo utiliza a matriz OD aérea como fonte da variável dependente
a ser estimada nos modelos.

\hypertarget{desenvolvimento-dos-modelos}{%
\section{Desenvolvimento dos
modelos}\label{desenvolvimento-dos-modelos}}

\hypertarget{dataset}{%
\subsection{\texorpdfstring{\emph{Dataset}}{Dataset}}\label{dataset}}

Talvez seja benéfico tratar da parte comum e de decisões que afetem os
dois modelos em uma seção unificada. Exemplo: variáveis geoeconômicas
utilizadas.

Retormar sobre matriz OD, dados do IPEA, quais variáveis são de
interesse.

\hypertarget{modelo-gravitacional-1}{%
\subsection{Modelo gravitacional}\label{modelo-gravitacional-1}}

Explicar as análises feitas para o desenvolvimento do modelo
gravitacional. Aqui, tratar da construção do modelo em si, calibragem
dos parâmetros, quantificação dos erros e validação da hipótese nula.

Falar da estimativa dos parâmetros pelo método de pseu máxima
verossimilhança de poisson (Poisson pseudo-maximum-likelihood), e porque
não linearizar com logartimos (vide \citet{martinez-zarzoso_log_2011}).

\hypertarget{modelo-economuxe9trico-regressuxe3o-linear-muxfaltipla}{%
\subsection{Modelo econométrico: regressão linear
múltipla}\label{modelo-economuxe9trico-regressuxe3o-linear-muxfaltipla}}

Inserir desenvolvimentos.

\hypertarget{modelo-de-machine-learning-rede-neural}{%
\subsection{\texorpdfstring{Modelo de \emph{machine learning}: rede
neural}{Modelo de machine learning: rede neural}}\label{modelo-de-machine-learning-rede-neural}}

Explicar as análises feitas para o desenvolvimento dos modelos de
machine learning.

\hypertarget{conclusuxf5es}{%
\section{Conclusões}\label{conclusuxf5es}}

Conclusão dos resultados obtidos.

%\showmatmethods
\showacknow


\bibliography{refs}
\bibliographystyle{jss}



\end{document}

